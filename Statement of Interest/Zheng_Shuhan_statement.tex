\documentclass{article}
\usepackage[utf8]{inputenc}
\usepackage[a4paper, margin=1.5in]{geometry}
\usepackage{fancyhdr}
\pagestyle{fancy}
\usepackage{graphicx}
\usepackage{float}
\usepackage[usestackEOL]{stackengine}
\usepackage{amsmath}
\usepackage{svrsymbols}
\graphicspath{{./image/}}
\usepackage{appendix}
\usepackage{csquotes}
\usepackage{listings}
\usepackage{url}
\usepackage[backend=biber, style=numeric]{biblatex}
\usepackage{hyperref}
\usepackage{indentfirst}


\addbibresource{SoI-blx.bib}
 
\title{UofT Astro SURP Statement of Interest}
\author{Shuhan Zheng (1007146154)}
\date{Janurary 2023}

\begin{document}

\maketitle

My interest in space goes back to my earliest memory of watching the launch of the Shenzhou-5 spacecraft with my grandfather. In the years after he passed, I turned an act of honoring his memory and legacy into a personal quest to be involved in humanity's combined efforts of exploring the final frontier. In particular, I prize my fascination with applying scientific knowledge to develop scientific research instruments. That fascination has been a primary driving factor in choosing my programs and career. 

Compared to many of my peers in physics and astronomy, I have more experience and passion for designing and building things. I played with my toys in elementary school by disassembling them and putting them back. In CEGEP, I joined and then led a student rocket club. In my first year at UofT, I assembled a team to compete in the Moon Society's Moon Base Design Contest. With me being the team leader and the first author of our design paper, my team became one of the ten finalists\footnote{\href{https://www.moonsociety.org/news/2021/03/10/announcement-of-winners-for-the-moon-societys-first-moon-base-design-contest/}{Announcement of Winners for the Moon Society's First Moon Base Design Contest} (Clickable link)}. Since September 2021, I have worked in the Space Systems division of the University of Toronto Aerospace Team (UTAT-SS). As a senior member, I improved upon an algorithm from the literature and adapted it to accommodate my team's unique camera design. I have also been instrumental in our development of a smile quantification and correction algorithm for our hyperspectral remote sensing camera. As a result, I became one of the authors of our team's design paper submitted to the Small Satellite Conference of 2022\footnote{\href{https://digitalcommons.usu.edu/smallsat/2022/all2022/88/}{FINCH: A Blueprint for Accessible and Scientifically Valuable Remote Sensing Satellite Missions} (Clickable link)}. Additionally, I enjoy challenging myself with projects\footnote{\href{https://github.com/ThisisShoo/Pet-Projects}{My GitHub page for my pet projects.} (Clickable link)} in my free time; sometimes, these projects are oriented toward solving real-world problems. 

One of the SURF projects that stood out to me was "Photonic Adaptive Optics," led by Dr. Momen Diab and Prof. Suresh Sivanandam. I want to participate in this project because it reminded me of several former projects of mine, in each of which I enjoyed 

% Below is draft
One of the SURF projects that stood out to me was "Photonic Adaptive Optics," led by Dr. Momen Diab and Prof. Suresh Sivanandam. I want to participate in this project because it reminded me of several projects I have been involved with. Also, upon reading Dr. Diab's \textit{Starlight coupling through atmospheric turbulence into few-mode fibres and photonic lanterns in the presence of partial adaptive optics correction}, I adored the elegance in calibrating an Adaptive Optics system using guide stars, rather than controlling the mirrors through a highly sophisticated atmospheric model.


\end{document}